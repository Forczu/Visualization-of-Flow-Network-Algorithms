% !TeX spellcheck = pl_PL
\chapter{Podsumowanie}
Celem niniejszej pracy było zrealizowanie wizualizacji działań na abstrakcyjnym modelu danych jakim są sieci przepływowe. Skupiono się na prezentacji algorytmów rozwiązujących problem wyszukiwania maksymalnego przepływu, jednak nie jest to jedyna rodzina algorytmów jaką można realizować dzięki temu modelowi. To najprostszy problem jaki można rozwiązać w sieci przepływowej, lecz najbardziej podstawowy i najchętniej opisywany w literaturze. Idee oraz pseudokody wybranych i zrealizowanych algorytmów zostały zawarte w rozdziale \ref{sec:analizaAlgorytmy}.\\\indent
Okazało się niemożliwym opisanej całej funkcjonalności oraz szczegółów implementacji jaką posiada aplikacja (m.in. sposobu łączenia rysunków wierzchołków i łuków w jedną całość), więc w specyfikacji wewnętrznej (rozdział \ref{chp:specWew}) skupiono się na opisaniu najistotniejszych elementów oprogramowania: reprezentacji sieci w pamięci komputera, zapewnieniu jej zgodności z teorią oraz sposobach realizacji algorytmów na niej. Ponadto przedstawiono zastosowane wzorce projektowe oraz możliwości rozszerzania aplikacji w rozdziale \ref{sec:solid}.\\\indent
Praca inżynierska była największym projektem programistycznym w trakcie studiów. Wymagała całego zakresu wiedzy, jaki pojawił się na programowaniu komputerów i inżynierii oprogramowania, a także wiele dodatkowego samokształcenia. Największym wyzwaniem było poprawne zaprojektowanie architektury aplikacji, która decydowała o całym późniejszym kierunku rozwoju.