% !TeX spellcheck = pl_PL
\chapter{Wstęp}
Sieć przepływowa jest abstrakcyjnym modelem danych, który jest wykorzystywany do rozwiązywania problemów związanych z przepływem produktów. Podobnie jak w klasycznym problemie producenta i konsumenta, istnieje w systemie źródło, gdzie produkty są tworzone, oraz ujście, gdzie są konsumowane. W sieci przepływowej zakłada się, że produkty są generowane i użytkowane w tym samym tempie, więc problemem nie jest synchronizacja, ale ilość produktów jakie można maksymalnie przesłać.\\\indent
Intuicyjnie można wyobrazić sobie sieć przepływową jako układ hydrauliczny, gdzie krawędziami grafu są rury o stałej średnicy, wierzchołki jako połączenia rur, a dwa z nich są wyszczególnione jako miejsce wlewu oraz wylewu cieczy. Średnica każdej rury określa jej przepustowość, czyli maksymalną ilość cieczy jaka może przez nią przepłynąć. Połączenia są jedynie rozgałęzieniami - nie gromadzą płynu, jedynie przekazują go dalej. Innymi słowy, ilość cieczy jaka wpływa do rozgałęzienia musi być równa ilości jaka z niej wypływa. Jest to ta sama zasada co pierwsze prawo Kirchhoffa dla przepływu prądu w obwodzie elektrycznym.\\\indent
Z siecią przepływową związany jest \textit{problem maksymalnego przepływu}, czyli największej wartości produktów, jaka może przepłynąć w danej sieci ze źródła do ujścia. W swojej pracy inżynierskiej zająłem się implementacją trzech algorytmów służących do znajdowania maksymalnego przepływu w sieci: Forda-Fulkersona, Dinica oraz MKM (Malhortra, Kumar i Mahaswari). Zaprojektowana przeze mnie aplikacja ma mieć charakter edukacyjny: student, który będzie korzystał z programu powinien móc utworzyć własną sieć (wedle swojego zamysłu bądź odwzorować przykład z książki) oraz wykonać na niej jeden z algorytmów. Aplikacja ma na celu umożliwienie mu zrozumienie działania algorytmu krok po kroku, pracę na wielu sieciach oraz zapisanie swojego postępu.

\begin{itemize}
	\item krótkie wprowadzenie (ok. 2 strony)
	\item przewodnik po pracy
	\item wstęp (i zakończenie) najłatwiej jest pisać na końcu
\end{itemize}