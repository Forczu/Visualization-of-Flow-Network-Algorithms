% !TeX spellcheck = pl_PL
\chapter{Wstęp}
Sieć przepływowa jest abstrakcyjnym modelem danych, który jest wykorzystywany do rozwiązywania problemów związanych z przepływem produktów. Podobnie jak w klasycznym problemie producenta i konsumenta, istnieje w systemie źródło, gdzie produkty są tworzone, oraz ujście, gdzie są konsumowane. W sieci przepływowej zakłada się, że produkty są generowane i użytkowane w tym samym tempie, więc problemem nie jest synchronizacja, ale ilość produktów jakie można maksymalnie przesłać.\\\indent
Intuicyjnie można wyobrazić sobie sieć przepływową jako układ hydrauliczny, gdzie krawędziami grafu są rury o stałej średnicy, wierzchołki są połączeniami rur, a dwa z nich są wyszczególnione jako miejsce wlewu oraz wylewu cieczy. Średnica każdej rury określa jej przepustowość, czyli maksymalną ilość cieczy jaka może przez nią przepłynąć. Połączenia są jedynie rozgałęzieniami - nie gromadzą płynu, jedynie przekazują go dalej. Innymi słowy, ilość cieczy jaka wpływa do rozgałęzienia musi być równa ilości jaka z niej wypływa. Jest to ta sama zasada co pierwsze prawo Kirchhoffa dla przepływu prądu w obwodzie elektrycznym.\\\indent
Z siecią przepływową związany jest \textit{problem maksymalnego przepływu}, czyli największej wartości produktów, jaka może przepłynąć w danej sieci ze źródła do ujścia. W ramach pracy inżynierskiej zaimplementowano trzy algorytmy służące do znajdowania maksymalnego przepływu w sieci: Forda-Fulkersona, Dinica oraz MKM (Malhortra, Kumar i Mahaswari). Zaprojektowana aplikacja ma mieć charakter edukacyjny: student, który będzie korzystał z programu powinien móc utworzyć własną sieć (wedle swojego zamysłu bądź odwzorować przykład z książki) oraz wykonać na niej jeden z algorytmów. Aplikacja ma na celu umożliwienie mu zrozumienie działania algorytmu krok po kroku, pracę na wielu sieciach oraz zapisanie swojego postępu.\\\indent
Aby praca mogła spełnić swój walor edukacyjny, jej celem nie były tylko implementacje gotowych algorytmów. W jej ramach został utworzony graficzny interfejs użytkownika, który umożliwia wygodne poruszanie się po aplikacji, tworzenie sieci i obserwację działania algorytmów. Proces wykonywania algorytmów został zobrazowany poprzez podświetlanie odpowiednich wierzchołków i łuków, wyświetlanie struktur pośrednich, usuwanie niepotrzebnych elementów oraz wypisywanie komentarzy. Jeżeli utworzona przez użytkownika sieć nie spełnia teoretycznych założeń sieci przepływowych, zostaje on o tym poinformowany wraz z wiadomością, jakie błędy popełnił. Ponadto użytkownik może konfigurować wygląd swoich sieci przepływowych, odczytywać i modyfikować zapisane sieci, a także zapisać sieć z obliczonym maksymalnym przepływem.\\\indent
W rozdziale drugim, \textbf{analizie tematu}, zostały przedstawione wszystkie najważniejsze pojęcia związane z teorią grafów i sieci przepływowych, o które została oparta praca. Zostały w nim również pokrótce omówione wszystkie algorytmy, jakie zostały zrealizowane w tej pracy, wraz z pseudokodami oraz wyjaśnieniami ich działania. W rozdziale trzecim znajdują się \textbf{wymagania}, gdzie przedstawiono idee jak aplikacja powinna wyglądać od strony użytkownika oraz jakie funkcjonalności powinna zawierać. Rozdział czwarty jest \textbf{specyfikacją zewnętrzną}, gdzie szczegółowo opisano jak należy korzystać z utworzonej aplikacji. W rozdziale piątym zawarto \textbf{specyfikację wewnętrzną} w której przedstawiono i omówiono najważniejsze fragmenty kodu związane z implementacją algorytmów. Pokazane wszystkie istotne struktury danych, zależności między klasami, zastosowane wzorce projektowe oraz pseudokody rozwiązań problemów, jakie były konieczne do poprawnego działania programu. Rozdział szósty to \textbf{testowanie i uruchamianie}, gdzie krótko zaprezentowano kilka modeli testowych oraz rozwiązywanie na nich problemu wyszukiwania maksymalnego przepływu. W rozdziale siódmym znajdują się \textbf{uwagi o przebiegu pracy}, gdzie zostały przedstawione najważniejsze problemy jakie pojawiły się w czasie rozwoju aplikacji i ich rozwiązania, a także potencjalne rozszerzenia. W rozdziale ósmym zawarto \textbf{podsumowanie} pracy inżynierskiej.
